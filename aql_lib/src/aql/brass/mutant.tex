\documentclass{article}
\usepackage[utf8]{inputenc}
\usepackage{amssymb}

\begin{document}

The source table will be unchanged from the original to the target schema.
All other target tables [hereafter referred to as "targets"]
must be mutually bijective.

The original cot_event and cot_event_position are bijective.
The bijection is made explicit in the schema by making cot_event_id both
the primary key and a foreign key.
\begin{verbatim}
CONSTRAINT cot_event_position_pkey PRIMARY KEY (cot_event_id)
CONSTRAINT cot_event_fk FOREIGN KEY (cot_event_id)
      REFERENCES takrpt.cot_event (id) MATCH SIMPLE]
\end{verbatim}

All remaining attributes from the original schema must
be presented in the targets.
The source_id foreign_key will be present in exactly one of the targets.
The primary keys for the targets will each be named 'id'
and are not specified in the perturbation message.
Each of the targets' primary keys are isomorphic with the
original cot_event.id and cot_event_position.cot_event_id primary keys.
These primary keys also serve as the foreign keys to all other targets.
This effectively forms an isomorphic clique between all targets.
This makes it appear that cot_event_position.cot_event_id
is dropped but it is not.

Currently the perturbation message shows the targets
and the attributes which they have been assigned.
Everything else is covered by the rules.

\end{document}
